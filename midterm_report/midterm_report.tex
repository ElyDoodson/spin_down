\documentclass[fleqn,usenatbib]{mnras}

% MNRAS is set in Times font. If you don't have this installed (most LaTeX
% installations will be fine) or prefer the old Computer Modern fonts, comment
% out the following line
% \usepackage{newtxtext,newtxmath}
% Depending on your LaTeX fonts installation, you might get better results with one of these:
% \usepackage{mathptmx}
% \usepackage{txfonts}

% Use vector fonts, so it zooms properly in on-screen viewing software
% Don't change these lines unless you know what you are doing
\usepackage[T1]{fontenc}

% Allow "Thomas van Noord" and "Simon de Laguarde" and alike to be sorted by "N" and "L" etc. in the bibliography.
% Write the name in the bibliography as "\VAN{Noord}{Van}{van} Noord, Thomas"
\DeclareRobustCommand{\VAN}[3]{#2}
\let\VANthebibliography\thebibliography
\def\thebibliography{\DeclareRobustCommand{\VAN}[3]{##3}\VANthebibliography}


%%%%% AUTHORS - PLACE YOUR OWN PACKAGES HERE %%%%%

% Only include extra packages if you really need them. Common packages are:
\usepackage{graphicx}
\graphicspath{ {C:/Users/elydo/Documents/Harvard/Midterm_report_images/} }	% Including figure files
\usepackage{amsmath}	% Advanced maths commands
\usepackage{amssymb}	% Extra maths symbols

%%%%%%%%%%%%%%%%%%%%%%%%%%%%%%%%%%%%%%%%%%%%%%%%%%

%%%%% AUTHORS - PLACE YOUR OWN COMMANDS HERE %%%%%

% Please keep new commands to a minimum, and use \newcommand not \def to avoid
% overwriting existing commands. Example:
\newcommand{\pcm}{\,cm$^{-2}$}	% per cm-squared

%%%%%%%%%%%%%%%%%%%%%%%%%%%%%%%%%%%%%%%%%%%%%%%%%%

%%%%%%%%%%%%%%%%%%% TITLE PAGE %%%%%%%%%%%%%%%%%%%

% Title of the paper, and the short title which is used in the headers.
% Keep the title short and informative.
% \title[Short title, max. 45 characters]{MNRAS \LaTeXe\ template --}
\title{PHYS6013 Midterm Report: A Machine Leanring Approach to Cool Star Spin-Down}

% The list of authors, and the short list which is used in the headers.
% If you need two or more lines of authors, add an extra line using \newauthor
\author[E. Doodson et al.]{
Ely Doodson,$^{1,2????,3???}$\thanks{E-mail: ely.doodson@cfa.harvard.edu}
Cecilia Garraffo,$^{2,3}$
Pavlos Protopapas$^{3}$
and Jeremy J. Drake$^{2}$
\\
% List of institutions
$^{1}$School of Physics and Astronomy, University of Southampton,
Southampton, SO17 1BJ, United Kingdom\\
$^{2}$Harvard-Smithsonian Center for Astrophysics, 60 Garden St, Cambridge, MA 02138, United States \\
$^{3}$Institute for Applied Computational Science, Harvard University, Cambridge, MA 02138, United States
}

% These dates will be filled out by the publisher
% \date{Accepted XXX. Received YYY; in original form ZZZ}

\pubyear{2020}

%imported for strikethrough
\usepackage{ulem}

\begin{document}
\label{firstpage}
\pagerange{\pageref{firstpage}--\pageref{lastpage}}
\maketitle

% Abstract of the paper
\begin{abstract}
	% This is a simple template for authors to write new MNRAS papers.
	% The abstract should briefly describe the aims, methods, and main results of the paper.
	% It should be a single paragraph not more than 250 words (200 words for Letters).
	% No references should appear in the abstract.
	Observations of young open clusters have shown a bimodal distribution in the rotation
	periods of cool stars.
	This bi-modality stems from stars having fast or slow rotation periods.
	The evolution of this trend through time suggests a fast transition from fast to slow rotating.
	Our current understanding of cool star spin down, through magnetic braking, accounts for the slow-rotators branch, while the fast rotators remain somewhat of a mystery.

	Our goal is to build a predictive probabilistic spin-down model that links the period of a starat any given mass and age.
	We use machine learning to predict the age at which each star transitions from fast to slow-rotation.
	Using a graphical model we translate the distribution of initial periods into a rotation period probability distribution for a given mass and age.
\end{abstract}

%%%%%%%%%%%%%%%%%%%%%%%%%%%%%%%%%%%%%%%%%%%%%%%%%%
%%%%%835wordsInTotalBeforeBody%%%%%%%%%%%%%%%
%%%%%%%%%%%%%%%%% BODY OF PAPER %%%%%%%%%%%%%%%%%%

\section{Introduction}
\sout{Stars born spinning, Over time they spin down with a mechanism.
	First pointed out by skumanich(feed back mechqnism so they converge(Faster they spin the LESS??? they lose), however fast branch escape thix for long period of time ) and studied further for hopes of a gyrochronological model.
	Magnetic breaking modeled and used as a method of spin down by Garraffo et al. 2017.
	Linked to magnetic field complexity.
	Dipole causes large arms that make for an effecient spin down.
	Viewing open clusters one can see the fast, slow and transitional rotators.
	Some evolution between the two that is UNKNOWN(?).}

Stars are born from the collapse of clouds made of dust and gas.
This cloud, though being made up of many molecules with their own random velocities, can be said to have an overall spin which, when the star collapses, will be the axis of rotation when the star is born.
However, this spin rate is not static and will descrease over a stars lifetime, assuming no companion or outside influence.

Cool stars are classified as having a convective envelope, meaning the outer-most layer of the star is moving.
They have a mass, $m \lesssim 1.3 \textup{M}_\odot$. 
Due to the convection, moving ionic material in the star generates massive magnetic fields which stretch in orders of stellar radii.
Ejected stellar material travels along these lines, forming large arms which effectively co-rotate with the star.
When material at the end of these arms breaks free, the loss in angular momentum (AM) is much grater than it would have been if the same material was lost at the stars surface.
This loss in AM causes the star to lessen its rotational period and spin down.
This is called \textit{magnetic breaking} and is a very effecient way for the star to spin down.

Open clusters(OC) are coeval groups of stars and, when obeserved in period and mass space, two distinct populations of fast and slow rotators can be seen after $\approx 10$Myrs; before this, stellar disc effects make it hard to model what is happening.
In addition there are  transitional stars between these two populations, which have a density less than the two populations, showing the time to move between the populations must be rapid.
This can be seen in Figure \ref{fig:slow_fast_transition}.
It was shown in \cite{}(Garraffo et al. 2017) that, when accounting for magnetic breaking and AM loss, this bimodal population could be seen with an evolved simulation of stars.
\begin{figure}
	\includegraphics[width = 0.5\textwidth]{slow_fast_transition.png}
	\caption[]{Plot of period vs mass(inverted) for Praesepe, with visulisation of the slow, fast and transitional rotators.}
	\label{fig:slow_fast_transition}
\end{figure}

\section{Methods and Observations}
\subsection{Data Reduction}

\sout{Collected from various papers stated in CITE BOOK, plus personal findings and unpublished data.
	Each cluster in different photometry and no one fits all conversion.
	Used MIST tracks to interpolate a mass from the given ages of the stars.

	FIGURE OF ALL CLUSTERS REDUCED
	MERGE CLUSTERS?}

Though there are lots of field stars with known masses and periods, the difficulty lies in having a correct value for their ages.
This makes OC, with known ages, ideal for collecting data on period and mass, which can be used readily in the expansion of gyrochronological models. 
The vast majority of OC data for this project was gathered from \cite{bookthingy}(PLANET BOOK et al. YEAR) as well as some new additions to already existing OC M37 \cite{chang}(Chang et al. 2017) \sout{and some unpublished data generously provided by Jason Curtis.}
Although all these catalogs contained values for period, mass was not often provided.
The magnitude and waveband(frequency maybe?) of light is corralated heavily with the mass of a star, so this is the parameter one observes when measuring the mass of a star. 
Each catalog presented their data in a different band of light, which make it difficult to convert them all to mass as different conversions were needed.
These conversions are complex for each band and vary greatly as a function of mass, with some conversions not covering our lowest masses.

A conversion was possible, however, it did not use the conventional functions to map photometry to mass. 
Instead I used the Modules for Experiments in Stellar Astrophysics (MESA) Isochrones and Stellar Tracks (MIST) tables \cite{MIST}(MISTTHING et al. 2004), which are simulations that provide information on the properties of stars, for a range of masses, evolved through time.
The converted OC can be seen in Figure \ref{fig:allclusters}
\begin{figure}
	\centering
	\includegraphics[width = 0.5\textwidth]{allclusters.png}
	\caption[]{Plot showing all converted OC and their respective ages}
	\label{fig:allclusters}
	
\end{figure}


These tables start with discrete mass steps (e.g 0.1, 0.15...1.35, 1.40~$\textup{M}_\odot$ etc).
These masses, evolving at differnt rates, are very likely to have a degeneracy in their photomerty, meaning they may cross each others "photometry path", and as a consequence the conversion was not as simple as choosing the two closest photometries and interpolating.
Instead I restricted the values of the various mass tracks to the clostest ages of that star and from this pool, interpolated between the two clostest photometries for that given age.
The discontinuity of the tracks, due to discrete time steps, means there will be an inherent error in choosing the pool of ages, this has not currently been addressed, however may be implemented into error propagation in a future model if a Bayesian network approach is used. 



\subsubsection{M37 shift?}
\sout{Perhaps due to metalicity}

\subsection{Unsupervised Clustering?}
\sout{Initial attempt to seperate the fast and slow rotators however problematic due to it not being a "two group" problem.
	Transitional stars need to be considered, otherwise subjecting the transition to a dirac delta.}

Our inital approach to the problem was to cluster the data into "fast" and "slow" rotators and fit a weighted polynomial regression to each of these groups.
We then cycle through each star in the cluster and assign it to the other group, if the overall fit is better with the star in the opposing group, it remained there, otherwise it was transfered back and the next star was assessed.
After each star was assessed, a new polynomial fit was generated for each cluster the process repeated until stars no longer changed groups.
To measure the fit, mean squared error(MSE), Equation \ref{eq:mse}, was minimised and reduced.
\begin{equation}
	\label{eq:mse}
	\textup{MSE} = \frac{1}{N}\sum_{i=1}^N \left(y_i - \hat{y}_i\right)^2
\end{equation}
where $y_i$ is the true value, and $\hat{y}_i$ is the predicted value.

A reduction in MSE means the predicted values are close to the true values and a good fit is being generated, if we minimise this MSE then we are rewarding the model with a fit closer to the true data.

\begin{figure}
	\includegraphics[width = 0.5\textwidth]{unsupervised_clustering.png}
	\caption{shows the results of unsupervised clustering of the "fast" rotators in blue, "slow" rotators in orange and their respective polynomial fits.}
	\label{fig:unsupervised_clustering}
\end{figure}

Figure \ref{fig:unsupervised_clustering} shows the results off this approach.
The fits generated were a polynomial of $3^\textup{rd}$ order for the blue "slow" rotators, and $1^\textup{st}$ order for the orange "fast" rotators.
As it can be seen, this approach does not generate an accurate representation of the two groups.
This is because the transition between these two clusters is effectively instantaneous, however, in reality this is not the case.
As we also wanted to try understand how these stars transitioned throughout their lifetime, this approach would not extract this kind of information and so clustering was deemed an uneffective method.

\subsection{Polynomial Ridge Regression}
\sout{slow rotators fit using a polynomial fit of order 4.
	Sigmoid function overlapped and optimised to change poly term on and off at switch point.

	FIGURE OF CURRENT FIT}

Our next approach was to fit a polynomial to some of the fast rotators and combine this polynomial with a sigmoid function to have the transition between the fast and slow rotators.
The method was to manually remove the "slow" and as many transitional rotators for each cluster as possible.
From this subset of stars I created 10 bins according to mass and reduced this subset to the central 80\% to remove outliers and and remaining transitional stars.
This reduced data set was then fit using ridge regression, which is a form of polynomial regression, with the addition of regularisation term, which stops the coefficients becoming too large.
This regularisation coefficient and the degree of polynomial was chosen via \textit{Cross Validation}.

splits, aintaining the data distribution ad the original.
then we do it for 10 splits

Cross validation is method to choose the best model from a selection, such as we have here with varying polynomial degree and regularisation term.
Your data is split into a number of random subsets which maintain the original data distribution.
One of these subsets is chosen as the validation set, and used to score your current model.
The remaining subsets are then used to train your model on.
The validation set is then changed and the same process repeated until every subset has been a validation set.
This ensures that all data has been used to train the models and to be validated on.
All these scores are then averaged to give a score for your model.
Figure \ref{fig:cross_validation} shows this for 10 subsets.

\begin{figure}
	\includegraphics[width = 0.5\textwidth]{cross_validation.png}
	\caption{shows how cross validation splits a data set and produces a score of that model.}
	\label{fig:cross_validation}
\end{figure}



\begin{figure}
	\includegraphics[width = 0.5\textwidth]{polyfit_twoclusters}
	\caption{shows M37 in blue and NGC6811 in orange with their respective best fits.}
	\label{fig:polyfit_twoclusters}
\end{figure}

In Figure \ref{fig:polyfit_twoclusters} I show the results of the process for two clusters of different ages.
It can be seen from the data and fits that the 450Myrs has allowed the stars on "slow" branch to reduce their spin further.
It can also be seen that there is a lack of a "fast" population for NGC6811 in the region $0.5~\textup{M}_\odot \leq m \leq 1.3~\textup{M}_\odot$.
This is because all these stars have had time to transition.

This optimised polynomial can then be combined with the sigmoid function, see Equation \ref{eq:sigmoid}, and minimised to produce our current model shown in Figure \ref{fig:model_fit}
\begin{figure}
	\includegraphics[width = 0.5\textwidth]{model_fit.png}
	\caption{shows an optimised fit using ridge regression and sigmoid for the OC Praesepe}
	\label{fig:model_fit}
\end{figure}
\begin{align}
	\centering
	\label{eq:sigmoid}
	\Phi\left(x\right) = \frac{1}{1 + \exp^{-x}}
\end{align}
The coefficients of these polynomials were thought to be the key to understanding the progression of the system.
However, it was found that there was no obivous trend to these coefficients through time.
The fit in Figure \ref{fig:model_fit} is able to predict the majority of the "slow" and "fast" rotator branch, however becomes problematic for the region which they overlap, which varies in size for different OC.
We think the cause of this degeneracy is caused by a spread of initial periods and is discussed further in Section \ref{sec:initial_period}.

A potential solution to this would be to calculate an error associated with each point of the fit.
A region of uncertainty around the prediction would be produced, which would be low if the data spread was minimal and high when approaching the overlapped area.

\subsection{The Importance of Initial Period, $P_i$}
\label{sec:initial_period}
\sout{Expand of the effect initial period and perhaps metalicity.
	Other parameters could allow for deeper understanding of the "overlap" sections of the open clusters

	FIGURE OF HOW INTIAL PERIOD CAUSES MULTIPLE LINES TO OVERLAP AND MAKE THE TRANSITION "BLURRY"}

We think the difficulty in these predictions of period, for a given age and mass, stem from the lack of information on the initial period distribution.
In a hypothetical OC, whose stars are born of a single initial rotation period, we think the distribution would look like one of the lines in Figure \ref{fig:initial_period}.
However in reality these observed OC are made up of a range of masses and initial periods, of which only the former can be measured.
The different lines in Figure \ref{fig:initial_period} represent how different values of initial period would look if evolved to the age of the cluster.
When taking an average of all these initial periods, we will get a distribution which looks like the observed OC, which has a degeneracy between the two branches.


\begin{figure}
	\includegraphics[width = 0.5\textwidth]{../AAS Poster Stuff/chiswitch_model.png}
	\label{fig:initial_period}
	\caption{shows what we hypothesise the OC star distribution would follow for 3 different initial periods, $\alpha, \beta~\&~\gamma$.}
\end{figure}

The best estimates for inital period of the system can be assumed to be the youngest clusters distribution that is no longer under the influence of disc effects, such as H Persei(NGC 869) in Figure \ref{fig:allclusters}.



\section{Future Work?}
\sout{Since there is overlap, 1 polynomial fit will give poor predictive results and without inital period is not purely deterministic to the degree we want.
	To remedy this a probabilistic model will be built that can be used to sample and generate a synthetic population of stars at a given age and a range of masses.}

\section*{Acknowledgements}

The Acknowledgements section is not numbered. Here you can thank helpful
colleagues, acknowledge funding agencies, telescopes and facilities used etc.
Try to keep it short.

%%%%%%%%%%%%%%%%%%%%%%%%%%%%%%%%%%%%%%%%%%%%%%%%%%

%%%%%%%%%%%%%%%%%%%%REFERENCES%%%%%%%%%%%%%%%%%%

%The best way to enter references is to use BibTeX:

\bibliographystyle{mnras}
\bibliography{example}

%%%%%%%%%%%%%%%%%APPENDICES%%%%%%%%%%%%%%%%%%%%%

\appendix
 
\section{Some extra material}

If you want to present additional material which would interrupt the flow of the main paper,
it can be placed in an Appendix which appears after the list of references.

%%%%%%%%%%%%%%%%%%%%%%%%%%%%%%%%%%%%%%%%%%%%%%%%%%


% Don't change these lines
\bsp	% typesetting comment
\label{lastpage}
\end{document}

% End of mnras_template.tex