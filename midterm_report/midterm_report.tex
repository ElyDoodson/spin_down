\documentclass[fleqn,usenatbib]{mnras}

% MNRAS is set in Times font. If you don't have this installed (most LaTeX
% installations will be fine) or prefer the old Computer Modern fonts, comment
% out the following line
% \usepackage{newtxtext,newtxmath}
% Depending on your LaTeX fonts installation, you might get better results with one of these:
% \usepackage{mathptmx}
% \usepackage{txfonts}

% Use vector fonts, so it zooms properly in on-screen viewing software
% Don't change these lines unless you know what you are doing
\usepackage[T1]{fontenc}

% Allow "Thomas van Noord" and "Simon de Laguarde" and alike to be sorted by "N" and "L" etc. in the bibliography.
% Write the name in the bibliography as "\VAN{Noord}{Van}{van} Noord, Thomas"
\DeclareRobustCommand{\VAN}[3]{#2}
\let\VANthebibliography\thebibliography
\def\thebibliography{\DeclareRobustCommand{\VAN}[3]{##3}\VANthebibliography}


%%%%% AUTHORS - PLACE YOUR OWN PACKAGES HERE %%%%%

% Only include extra packages if you really need them. Common packages are:
\usepackage{graphicx}
\graphicspath{ {C:/Users/elydo/Documents/Harvard/Midterm_report_images/} }	% Including figure files
\usepackage{amsmath}	% Advanced maths commands
\usepackage{amssymb}	% Extra maths symbols

%%%%%%%%%%%%%%%%%%%%%%%%%%%%%%%%%%%%%%%%%%%%%%%%%%

%%%%% AUTHORS - PLACE YOUR OWN COMMANDS HERE %%%%%

% Please keep new commands to a minimum, and use \newcommand not \def to avoid
% overwriting existing commands. Example:
\newcommand{\pcm}{\,cm$^{-2}$}	% per cm-squared

%%%%%%%%%%%%%%%%%%%%%%%%%%%%%%%%%%%%%%%%%%%%%%%%%%

%%%%%%%%%%%%%%%%%%% TITLE PAGE %%%%%%%%%%%%%%%%%%%

% Title of the paper, and the short title which is used in the headers.
% Keep the title short and informative.
% \title[Short title, max. 45 characters]{MNRAS \LaTeXe\ template --}
\title{PHYS6013 Midterm Report: A Machine Learning Approach to Cool Star Spin-Down}

% The list of authors, and the short list which is used in the headers.
% If you need two or more lines of authors, add an extra line using \newauthor
% \author[E. Doodson et al.]{
% Ely Doodson,$^{1,2,3}$\thanks{E-mail: ely.doodson@cfa.harvard.edu}
% Cecilia Garraffo,$^{2,3}$
% Pavlos Protopapas$^{3}$
% and Jeremy J. Drake$^{2}$
% \\
% % List of institutions
% $^{1}$School of Physics and Astronomy, University of Southampton,
% Southampton, SO17 1BJ, United Kingdom\\
% $^{2}$Harvard-Smithsonian Center for Astrophysics, 60 Garden St, Cambridge, MA 02138, United States \\
% $^{3}$Institute for Applied Computational Science, Harvard University, Cambridge, MA 02138, United States
% }
\author[E. Doodson et al.]{
Ely Doodson,
 Supervisor: Cecilia Garraffo \\
Additional supervision from: Pavlos Protopapas,
and Jeremy J. Drake
% \\
% % List of institutions
% $^{1}$School of Physics and Astronomy, University of Southampton,
% Southampton, SO17 1BJ, United Kingdom\\
% $^{2}$Harvard-Smithsonian Center for Astrophysics, 60 Garden St, Cambridge, MA 02138, United States \\
% $^{3}$Institute for Applied Computational Science, Harvard University, Cambridge, MA 02138, United States
}

% These dates will be filled out by the publisher
% \date{Accepted XXX. Received YYY; in original form ZZZ}

\pubyear{2020}

%imported for strikethrough
\usepackage{ulem}
\usepackage{float}

\begin{document}
\label{firstpage}
\pagerange{\pageref{firstpage}--\pageref{lastpage}}
\maketitle

% Abstract of the paper
\begin{abstract}
	% This is a simple template for authors to write new MNRAS papers.
	% The abstract should briefly describe the aims, methods, and main results of the paper.
	% It should be a single paragraph not more than 250 words (200 words for Letters).
	% No references should appear in the abstract.
	Observations of young open clusters have shown a bimodal distribution in the rotation
	periods of cool stars.
	This bi-modality stems from stars having fast or slow rotation periods.
	The evolution of this trend through time suggests a fast transition from fast to slow rotating.
	Our current understanding of cool star spin down, through magnetic braking, accounts for the slow-rotators branch, while the fast rotators remain somewhat of a mystery.

	Our goal is to build a predictive probabilistic spin-down model that links the period of a starat any given mass and age.
	We use machine learning to predict the age at which each star transitions from fast to slow-rotation.
	Using a graphical model we will translate the distribution of initial periods into a rotation period probability distribution for a given mass and age.
\end{abstract}

%%%%%%%%%%%%%%%%%%%%%%%%%%%%%%%%%%%%%%%%%%%%%%%%%%
%%%%%835wordsInTotalBeforeBody%%%%%%%%%%%%%%%
%%%%%%%%%%%%%%%%% BODY OF PAPER %%%%%%%%%%%%%%%%%%

\section{Introduction}
Stars are born from the collapse of clouds made of dust and gas.
Such a cloud, though being made up of many molecules with their own random velocities, can be said to be spinning in one direction overall.
Some of this angular momentum is sequestered in newly-formed stars which consequently rotate.
However, their rotation rates are not static and decrease over a stars lifetime as a result of angular momentum lost through stellar winds.

Cool stars, which are those classified as having a convective envelope (a turbulent layer in which convection carries a significant fraction of the outward energy flow) and a mass $\lesssim 1.3 \textup{M}_\odot$, have an interior magnetic dynamo powered by rotation which drives a magnetized wind. Ejected stellar material travels along magnetic field lines, effectively co-rotating with the star out to several stellar radii and carrying away much more angular momentum (AM) than if the material was lost at the stellar surface. 

The rate at which a star rotates drives the stellar stength of this magnetic breaking, the faster a star rotates, the stronger its magnetic dynamo and therefore the more prominant the winds that slow the star.
The resulting change in stellar rotation rate can in principle be used to estimate the age of cool stars, a technique called "gyrochronology".
The self-regulating magnetic breaking mechanism was expected to cause all the stars in an evolving population to converge together and is why gyrochronology works for older stars: they have had time to converge to the regime one can model.
However, when observing groups of coeval stars in open clusters (OC) with ages $< 1Gyrs$, a branch of fast rotating stars is also observed, see Figure \ref{fig:slow_fast_transition}.
This fast rotators branch is a mystery and is not encompassed by current gyrochronological models, leaving a gap in their predictive capabilities.
This work hopes to fill this gap.

% Cool stars are classified as having a convective envelope, meaning the outer-most layer of the star is moving.
% They have a mass, $m \lesssim 1.3 \textup{M}_\odot$. 
% Due to the convection, moving ionic material in the star generates massive magnetic fields which stretch in orders of stellar radii. (SOMEONE I NEED TO CITE?)
% Ejected stellar material travels along these lines, forming large arms which effectively co-rotate with the star.
% When material at the end of these arms breaks free, the loss in angular momentum (AM) is much grater than if the same material was lost at the stars surface.
% This loss in AM causes the star to lessen its rotation period and spin down.
% This is called \textit{magnetic breaking} and is a very effecient way for a star to spin down. (CITE HERE ALSO??? WHO?)

\begin{figure}
	\includegraphics[width = 0.5\textwidth]{slow_fast_transition.png}
	\caption[]{Plot of rotation period vs mass for the Praesepe young open cluster (age $\approx 790$ Myrs), with visulisation of the slow, fast and transitional rotators.}
	\label{fig:slow_fast_transition}
\end{figure}

Understanding the evolution of stellar rotation is crucial for understanding stellar magnetic activity as the magnetic dynamo generated by stellar rotation is responsible for many observed stellar characteristics such as star spots, UV/X-ray chromospheric and coronal emission, and the aforementioned magnetised winds driving stellar spin down.
So far our understanding of these magnetised winds is purely derived from studies of our own Sun so our understanding of their origin is therefore incomplete; stellar spin-down provides a powerful probe of the physics of these winds.
Stellar activity and wind create an energetic photon and particle radiation environment that has significant impact on the formation of planets and their subsequent atmospheric evolution: which is important for close-in planets, that constitute the vast majority of detected rocky planets orbiting in the "habitable zone".
Understanding this will be crucial to the development of the rising field of exoplanet habitability.

It was shown in \cite{Garraffo_2018} that, when accounting for the effects of surface magnetic field complexity on AM loss, the bimodal rotation population could be reproduced in simulations of stellar spin down.
This was an analytical approach to the problem, however, with a wealth of new data from Kepler, K2 and TESS, this has provided the opportunity to approach gyrochronology from a more data-driven approach and form an even greater understanding of the parameters that constrain such systems.
The aim of this project is to utilize this growing database of observed stellar rotation periods and new numerical methods of machine learning to make a model capable of predicting a star's period, for a given mass and age.

\section{Observations}
\subsection{Data Reduction}
OC are coeval groups of stars and, when observed in period and mass space, two distinct populations of fast and slow rotators can be seen after $\approx 10$Myrs; before this time, stars are thought to be magnetically coupled with their stellar discs which modulate stellar rotation through a process known as "disc locking". 
In addition there are  transitional stars between these two populations, which are less numerous, showing that the time to move between the populations must be rapid.
This can be seen in Figure \ref{fig:slow_fast_transition}.

Though there are lots of field stars with known masses and periods, the difficulty lies in having a correct value for their ages.
This makes OC with known ages, observed by Kepler, K2 and TESS, ideal for collecting data on period and mass.

Stellar rotation periods for OC were gathered from \cite{beuther2014protostars} and combined with our new additions to already existing OC measurements, such as for the M37 cluster \cite{chang}.
Although all these catalogs contained values for period, mass was not often provided.
Stellar masses were derived from catalogued multi-band photometry converted using stellar model photometric predictions. 

I used the Modules for Experiments in Stellar Astrophysics (MESA) Isochrones and Stellar Tracks (MIST) tables from \cite{Choi_2016}, which are simulations that provide information on the properties of stars, for a range of masses, evolved through time.
I have obtained masses for all OC stars in our master catalogue and display them in Figure \ref{fig:allclusters}.

\begin{figure}
	\centering
	\includegraphics[width = 0.5\textwidth]{allclusters.png}
	\caption[]{Plot showing all converted OC and their respective ages}
	\label{fig:allclusters}
\end{figure}

\sout{These tracks start at discrete mass steps (e.g 0.1, 0.15...1.35, 1.40~$\textup{M}_\odot$ etc).
These masses, evolving at different rates, are very likely to have a degeneracy in their photometry, meaning one star can have the same photometry value as another at a different age and mass, and as a consequence the conversion was not as simple as choosing the two closest photometries and interpolating.
Instead I restricted the values of the various mass tracks to the closest ages of that star and from this pool, interpolated between the two closest photometries for that given age.
The discontinuity of the tracks, due to discrete time steps, means there will be an inherent error in choosing the pool of ages, this has not currently been addressed, however may be implemented into error propagation in a future model if a Bayesian network approach is used.} TOO MUCH DETAIL?!?!?!

\section{Analysis Methods}
The methods laid out in this section are aimed at attaining a predicted period as a function of mass and age.
\subsection{Unsupervised Clustering}
Our initial approach to the problem was to use \textit{Unsupervised Clustering}.
Clustering is the process of grouping data of similar properties together such as stars on one or the other rotation branch.
The process being unsupervised means the data did not have a "correct" label telling us which group it belongs to, which one can use to reinforce and encourage certain clustering.

This clustering would split the data into "fast" and "slow" rotators and I could then fit a polynomial regression to each of these groups.
A polynomial regression is like linear regression, where you fit a straight line is fitted to the data $\hat{y} = b_0 + b_1x$, but with a polynomial you allow $n$ more higher order terms such as $...+ b_2x^2 + b_3x^3+b_4x^4+...+b_nx^n $ to be considered such that
\begin{equation}
	\label{eq:poly}
	\hat{y} = b_0 + \sum_{j = 1}^p b_jx^j
\end{equation}
This allows, $\hat{y}$, to be a closer approximation to the true value, $y$, because the higher order terms allow the regression to change more rapidly, fitting the data more closely.

After a polynomial regression of the two groups, I cycle through each star in a cluster and assign it one or the other of the groups.
After each star was assessed, a new polynomial fit was generated for each cluster and the process repeated until stars no longer changed groups.
To measure the score, the mean squared error(MSE), Equation \ref{eq:mse}, was minimised and reduced.
% \begin{align}a
\begin{gather}
	\begin{aligned}
		\label{eq:mse}
		\textup{MSE} &= \frac{1}{n}\sum_{i=1}^n \left(y_i - \hat{y}_i\right)^2
	\end{aligned}
	\\
	\begin{aligned}
		\textup{and substituting in Equation \ref{eq:poly}}\nonumber
	\end{aligned}
	\\
	\begin{aligned}
		\textup{MSE} = \frac{1}{n}\sum_{i = 1}^n \left(y_i - \left(b_0 + \sum_{j = 1}^p b_jx_i^j\right) \right)^2
	\end{aligned}
\end{gather}
where $y_i$ is the true value, and $\hat{y}_i$ is the predicted value.
% \begin{gather}
% 	\begin{split}
% 	  m &=n \\
% 		&=o \\
% 		&=p \\
% 		&=q
% 	\end{split}\\
% 	\begin{aligned}
% 	  a &=b + 1 \\
% 	  c &=d \\
% 	  e &=f \\
% 	  g &=h
% 	\end{aligned}
% \end{gather}
A reduction in MSE means the predicted values are closer to the true values and a better fit is being generated.
Minimising this MSE rewards the model that fits closer to the true data.

\begin{figure}
	\includegraphics[width = 0.5\textwidth]{unsupervised_clustering.png}
	\caption{The results of unsupervised clustering of Praesepe in which stars are assigned to either the fast or slow rotating groups. The "fast" rotators are shown in orange, "slow" rotators are shown in orange and their respective polynomial fits.}
	\label{fig:unsupervised_clustering}
\end{figure}

Figure \ref{fig:unsupervised_clustering} shows the results of this approach.
The fits generated were a polynomial $\mathcal{O}(x^3)$ for the blue "slow" rotators, and $\mathcal{O}(x)$ for the orange "fast" rotators.
As can be seen, this approach does not generate an accurate representation of the two groups.
This is because this modelling technique splits the data into two groups, which will have an instantaneous transition between them, however, in reality this is not the case, as can be seen from the large transitional region in \ref{fig:slow_fast_transition}.
% This is because the modelled transition between these two clusters is effectively instantaneous, however, in reality this is not the case.
We also wanted to try to understand how these stars transitioned and since this approach would not extract this kind of information clustering was deemed an insufficient method.

\subsection{Polynomial Ridge Regression}
Our next approach was to fit a polynomial to a portion of the "slow" rotators and combine this polynomial with a sigmoid function($\Phi\left(x\right)$), shown in Equation \ref{eq:sigmoid} below, to represent the transition between the fast and slow rotators.
The sigmoid function maps any value of $x$ to the range $ 0 \leq \Phi\left(x\right) \leq 1$, and can be thought of as switching between the two lines in Figure \ref{fig:slow_fast_transition}.

The method consisted of isolating just the slow rotators population for each cluster.
In order to do this I binned the data according to mass and selected the closest 80\% of the data to the mean to remove outliers and remaining transitional stars.
This reduced data set was then fit using {\textit{Ridge Regression}, which scores using MSE, Equation \ref{eq:mse}, with the addition of a regularisation term, $\lambda$, which stops the coefficients becoming too large, shown in Equation \ref{eq:ridge}.
This regularisation coefficient, $\lambda$, and the degree of polynomial, $p$, was chosen via \textit{Cross Validation}.

\begin{equation}
	\label{eq:ridge}
	\textup{MSE} = \frac{1}{n}\sum_{i = 1}^n \left(y_i - b_0 - \sum_{j = 1}^p b_jx_i^j \right)^2 + \lambda \sum_{j = 1}^p b^2_j
\end{equation}

Cross validation is a method to choose the best model from a selection, such as we have here with varying polynomial degree and regularisation term.
For each model, the data is split into a number of random subsets which maintain the original data distribution.
One of these subsets is chosen as the validation set, and used to score the current model.
The remaining subsets are then used to train the model on.
The validation set used is then changed and the same process repeated until every subset has been a validation set.
This ensures that all data has been used to train and validate the models.
All these scores are then averaged to give a performance for the model.
Figure \ref{fig:cross_validation} shows this for 5 subsets.

\begin{figure}
	\includegraphics[width = 0.5\textwidth]{kfold_crossval.png}
	\caption{An illustration of how cross validation splits a data set and produces a score of that model, from \protect\cite{CrossVal}.}
	\label{fig:cross_validation}
\end{figure}



\begin{figure}
	\includegraphics[width = 0.5\textwidth]{polyfit_twoclusters}
	\caption{The rotation period as a function of stellar mass for stars in  M37 in blue and NGC6811 in orange with their respective minimised polynomial fits.}
	\label{fig:polyfit_twoclusters}
\end{figure}

In Figure \ref{fig:polyfit_twoclusters} I show the results of only the polynomial ridge regression for two clusters of different ages.
It can be seen from the data and fits that the 450Myr difference has allowed the stars on the "slow" branch to reduce their spin further.
A lack of the "fast" rotator population for NGC6811 can be seen in the region $0.5~\textup{M}_\odot \leq m \leq 1.3~\textup{M}_\odot$.
This is because the higher mass stars transition first and have had time to transition.

This optimised polynomial can then be combined with the sigmoid function($\Phi\left(x\right)$), as seen in Equation \ref{eq:poly+sigmoid}, and minimised to produce our current model shown in Figure \ref{fig:model_fit}
\begin{figure}
	\includegraphics[width = 0.5\textwidth]{model_fit.png}
	\caption{The result of polynomial ridge regression minimised on Praesepe.}
	\label{fig:model_fit}
\end{figure}
\begin{align}
	\label{eq:poly+sigmoid}
	\textup{Period} = \left(b_0 + b_1m + b_2m^2...b_im^j\right)*\Phi\left(x\right) 
\end{align}
\begin{align}
	\label{eq:sigmoid}
	\Phi\left(x\right) = \frac{1}{1 + \exp^{-x}}
\end{align}
The coefficients of these polynomials were thought to be the key to understanding the progression of the system.
However, it was found that there was no obvious trend to these coefficients through time.
The fit in Figure \ref{fig:model_fit} is able to predict the majority of the "slow" and "fast" rotator branch, however, it becomes problematic for the region in which they overlap, which varies in size for different OC.
We think this degeneracy is caused by a spread of initial periods and is discussed further in Section \ref{sec:initial_period}.

\subsection{The Importance of Initial Period, $P_i$} \label{sec:initial_period}
\begin{figure}
	\includegraphics[width = 0.5\textwidth]{../AAS Poster Stuff/chiswitch_model.png}
	\caption{Theoretical lines that Praesepe data would follow if it was born with an initial period equal to a constant e.g. Constant1/Constant2/Constant3}
	\label{fig:spread_initial_period}
\end{figure}
We think the difficulty in these predictions of period, for a given age and mass, stem from the lack of information on the initial period distribution.
In a hypothetical OC, whose stars are born with the same constant initial rotation period, we think the stellar distribution would follow one of the lines in Figure \ref{fig:spread_initial_period}.
However, in reality these observed OC are made up of a range of masses and initial periods, of which only the former can be measured.
% The different lines in Figure \ref{fig:spread_initial_period} represent theoretical distributions, provided different values of initial period.
When taking an average of all these initial periods, we will get a distribution which looks like an observed OC, which has the degeneracy between the two branches due to the distribution of initial periods when the stellar population formed.

To try and address this, an estimate for initial period distribution can be made from the youngest cluster's distribution, that is no longer under the influence of disc effects, such as H Persei (h\_per) in Figure \ref{fig:allclusters}.

\section{Future Work?}
This paragraph goes here somewhere.
A potential solution to the degree would be to calculate an error associated with each point of the fit.
A region of uncertainty around the prediction would be produced, which would be low if the data spread was minimal and high when approaching the overlapped area.

The degeneracy discussed in previous sections is a large motivator to approach this model probabilistically.
Instead of generating a polynomial regression and modelling its evolution as a function of cluster age, a heat map of the likelihood of a star being in that spot for a given age and mass could be generated.
One could then sample stars of random masses for a given age to generate a synthetic population of stars, whos distribution looks like that of an observed OC.
% \section*{Acknowledgements}
% My amazing supervisors $<3$

% The Acknowledgements section is not numbered. Here you can thank helpful
% colleagues, acknowledge funding agencies, telescopes and facilities used etc.
% Try to keep it short.

%%%%%%%%%%%%%%%%%%%%%%%%%%%%%%%%%%%%%%%%%%%%%%%%%%

%%%%%%%%%%%%%%%%%%%%REFERENCES%%%%%%%%%%%%%%%%%%

%The best way to enter references is to use BibTeX:

\bibliographystyle{mnras}
\bibliography{references}

%%%%%%%%%%%%%%%%%APPENDICES%%%%%%%%%%%%%%%%%%%%%

% \appendix
 
% \section{Some extra material}

% If you want to present additional material which would interrupt the flow of the main paper,
% it can be placed in an Appendix which appears after the list of references.

%%%%%%%%%%%%%%%%%%%%%%%%%%%%%%%%%%%%%%%%%%%%%%%%%%


% Don't change these lines
\bsp	% typesetting comment
\label{lastpage}
\end{document}

% End of mnras_template.tex