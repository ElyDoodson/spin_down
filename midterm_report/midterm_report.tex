\documentclass[fleqn,usenatbib]{mnras}

% MNRAS is set in Times font. If you don't have this installed (most LaTeX
% installations will be fine) or prefer the old Computer Modern fonts, comment
% out the following line
\usepackage{newtxtext,newtxmath}
% Depending on your LaTeX fonts installation, you might get better results with one of these:
\usepackage{mathptmx}
\usepackage{txfonts}

% Use vector fonts, so it zooms properly in on-screen viewing software
% Don't change these lines unless you know what you are doing
\usepackage[T1]{fontenc}

% Allow "Thomas van Noord" and "Simon de Laguarde" and alike to be sorted by "N" and "L" etc. in the bibliography.
% Write the name in the bibliography as "\VAN{Noord}{Van}{van} Noord, Thomas"
\DeclareRobustCommand{\VAN}[3]{#2}
\let\VANthebibliography\thebibliography
\def\thebibliography{\DeclareRobustCommand{\VAN}[3]{##3}\VANthebibliography}


%%%%% AUTHORS - PLACE YOUR OWN PACKAGES HERE %%%%%

% Only include extra packages if you really need them. Common packages are:
\usepackage{graphicx}	% Including figure files
\usepackage{amsmath}	% Advanced maths commands
\usepackage{amssymb}	% Extra maths symbols

%%%%%%%%%%%%%%%%%%%%%%%%%%%%%%%%%%%%%%%%%%%%%%%%%%

%%%%% AUTHORS - PLACE YOUR OWN COMMANDS HERE %%%%%

% Please keep new commands to a minimum, and use \newcommand not \def to avoid
% overwriting existing commands. Example:
\newcommand{\pcm}{\,cm$^{-2}$}	% per cm-squared

%%%%%%%%%%%%%%%%%%%%%%%%%%%%%%%%%%%%%%%%%%%%%%%%%%

%%%%%%%%%%%%%%%%%%% TITLE PAGE %%%%%%%%%%%%%%%%%%%

% Title of the paper, and the short title which is used in the headers.
% Keep the title short and informative.
% \title[Short title, max. 45 characters]{MNRAS \LaTeXe\ template --}
\title{PHYS6013 Midterm Report: A Machine Leanring Approach to Cool Star Spin-Down}

% The list of authors, and the short list which is used in the headers.
% If you need two or more lines of authors, add an extra line using \newauthor
\author[E. Doodson et al.]{
Ely Doodson,$^{1,2????,3???}$\thanks{E-mail: ely.doodson@cfa.harvard.edu}
Cecilia Garraffo,$^{2,3}$
Pavlos Protopapas$^{3}$
and Jeremy J. Drake$^{2}$
\\
% List of institutions
$^{1}$School of Physics and Astronomy, University of Southampton,
Southampton, SO17 1BJ, United Kingdom\\
$^{2}$Harvard-Smithsonian Center for Astrophysics, 60 Garden St, Cambridge, MA 02138, United States \\
$^{3}$Institute for Applied Computational Science, Harvard University, Cambridge, MA 02138, United States
}

% These dates will be filled out by the publisher
% \date{Accepted XXX. Received YYY; in original form ZZZ}

\pubyear{2020}

%imported for strikethrough
\usepackage{ulem}

\begin{document}
\label{firstpage}
\pagerange{\pageref{firstpage}--\pageref{lastpage}}
\maketitle

% Abstract of the paper
\begin{abstract}
	% This is a simple template for authors to write new MNRAS papers.
	% The abstract should briefly describe the aims, methods, and main results of the paper.
	% It should be a single paragraph not more than 250 words (200 words for Letters).
	% No references should appear in the abstract.
	Observations of young open clusters have shown a bimodal distribution in the rotation
	periods of cool stars.
	This bi-modality stems from stars having fast or slow rotation periods.
	The evolution of this trend through time suggests a fast transition from fast to slow rotating.
	Our current understanding of cool star spin down, through magnetic braking, accounts for the slow-rotators branch, while the fast rotators remain somewhat of a mystery.

	Our goal is to build a predictive probabilistic spin-down model that links the period of a starat any given mass and age.
	We use machine learning to predict the age at which each star transitions from fast to slow-rotation.
	Using a graphical model we translate the distribution of initial periods into a rotation period probability distribution for a given mass and age.
\end{abstract}

%%%%%%%%%%%%%%%%%%%%%%%%%%%%%%%%%%%%%%%%%%%%%%%%%%
%%%%%835wordsInTotalBeforeBody%%%%%%%%%%%%%%%
%%%%%%%%%%%%%%%%% BODY OF PAPER %%%%%%%%%%%%%%%%%%

\section{Introduction}
Stars are born from the collapse of clouds made of dust and gas.
This cloud, though being made up of many molecules with their own random velocities, can be said to have an overall spin which, when the star collapses, will be the axis of rotation when the star is born.
However, this spin rate is not constant and will descrease over a stars lifetime, assuming no companion or outside influence.

Cool stars are classified as having a conventive envelope, meaning the outer-most layer of the star is moving.
Due to this movement, ionic material in the star is allowed to generate massive magnetic fields which stretch in orders of stellar radii.
Ejected stellar material travels along these lines, forming large arms which effectively co-rotate with the star.
When material at the end of these arms breaks free, the loss in angular momentum (AM) is much grater than it would have been if the same material was lost at the stars surface.
This loss in AM causes the star to lessen its rotational period and spin down.
This is called \textit{magnetic breaking} and is a very effecient way for the star to spin down.

Open clusters(OC) are coeval groups of stars and, when obeserved in period and mass space, two distinct populations of fast and slow rotators can be seen after $\approx 10$Myrs; before this, stellar disc effects make it hard to model what is happening.
In addition there are, to a lesser extent, transitional stars between these two populations, showing the time to move between the populations must be rapid.
It was shown in \cite{}(Garraffo et al. 2017) that, when accounting for magnetic breaking and AM loss, this bimodal population could be seen with an evolved simulation of stars.

(((((((())))))))

\sout{Stars born spinning, Over time they spin down with a mechanism.
	First pointed out by skumanich(feed back mechqnism so they converge(Faster they spin the LESS??? they lose), however fast branch escape thix for long period of time ) and studied further for hopes of a gyrochronological model.
	Magnetic breaking modeled and used as a method of spin down by Garraffo et al. 2017.
	Linked to magnetic field complexity.
	Dipole causes large arms that make for an effecient spin down.
	Viewing open clusters one can see the fast, slow and transitional rotators.
	Some evolution between the two that is UNKNOWN(?).}

\section{Methods and Observations}
\subsection{Data Reduction}

\sout{Collected from various papers stated in CITE BOOK, plus personal findings and unpublished data.
	Each cluster in different photometry and no one fits all conversion.
	Used MIST tracks to interpolate a mass from the given ages of the stars.

	FIGURE OF ALL CLUSTERS REDUCED
	MERGE CLUSTERS?}

Though there are lots of field stars with known masses and periods, the difficulty lies in having a correct value for their ages.
This makes OC, with a known age, ideal for the data used in this modelling process if we know the period and mass of the stars contained.
The vast majority of OC data for this project was gathered from \cite{bookthingy}(PLANET BOOK et al. YEAR) as well as some new additions to already existing OC M37 \cite{chang}(Chang et al. 2017) and some unpublished data generously provided by Jason Curtis.
Although all these catalogs contained values for period, mass was not often provided.
Instead, mass was effectively given by the photometry values of each star, with each catalog providing different bands for their stars.
This made it difficult to convert all the stars to mass as different conversion are needed for different bands and different mass ranges, with some conversions not stretching as low as the lowest mass.

A conversion was possible, however, it did not use conventional functions to map photometry to mass. 
Instead I used the MESA Isochrone and Stellar Track(MIST) tables \cite{MIST}(MISTTHING et al. 2004), which are simulations that provide information on the properties of stars, for a range of masses, evolved through time.

These tables start with discrete mass steps(e.g 0.1, 0.15...1.35, 1.40~$\textup{M}_\odot$ etc).
These masses, evolving at differnt rates, are very likely to have a degeneracy in their photomorty, meaning they may cross each others "photometry path" and as a consequence the conversion was not as simple as choosing the two closest photometries and interpolating.
Instead I had to restrict the available pool of photometries, based on the closest ages to that star, and from that pool choose the clostest photometry to interpolate between.
The discontinuity of the tracks, due to discrete time steps, means there will be an inherent error in choosing the pool of ages, this has not currently been addressed, however may be implemented into error propagation in a future model if a Bayesian network approach is used. 


\subsubsection{M37 shift?}
\sout{Perhaps due to metalicity}

\subsection{Unsupervised Clustering?}
\sout{Initial attempt to seperate the fast and slow rotators however problematic due to it not being a "two group" problem.
	Transitional stars need to be considered, otherwise subjecting the transition to a dirac delta.}


\subsection{Polynomial Ridge Regression}
\sout{slow rotators fit using a polynomial fit of order 4.
	Sigmoid function overlapped and optimised to change poly term on and off at switch point.

	FIGURE OF CURRENT FIT}

\subsection{Initial Period and other parameters}
\sout{Expand of the effect initial period and perhaps metalicity.
	Other parameters could allow for deeper understanding of the "overlap" sections of the open clusters

	FIGURE OF HOW INTIAL PERIOD CAUSES MULTIPLE LINES TO OVERLAP AND MAKE THE TRANSITION "BLURRY"}



\section{Future Work?}
\sout{Since there is overlap, 1 polynomial fit will give poor predictive results and without inital period is not purely deterministic to the degree we want.
	To remedy this a probabilistic model will be built that can be used to sample and generate a synthetic population of stars at a given age and a range of masses.}

\section*{Acknowledgements}

The Acknowledgements section is not numbered. Here you can thank helpful
colleagues, acknowledge funding agencies, telescopes and facilities used etc.
Try to keep it short.

%%%%%%%%%%%%%%%%%%%%%%%%%%%%%%%%%%%%%%%%%%%%%%%%%%

%%%%%%%%%%%%%%%%%%%%REFERENCES%%%%%%%%%%%%%%%%%%

%The best way to enter references is to use BibTeX:

\bibliographystyle{mnras}
\bibliography{example}

%%%%%%%%%%%%%%%%%APPENDICES%%%%%%%%%%%%%%%%%%%%%

\appendix

\section{Some extra material}

If you want to present additional material which would interrupt the flow of the main paper,
it can be placed in an Appendix which appears after the list of references.

%%%%%%%%%%%%%%%%%%%%%%%%%%%%%%%%%%%%%%%%%%%%%%%%%%


% Don't change these lines
\bsp	% typesetting comment
\label{lastpage}
\end{document}

% End of mnras_template.tex